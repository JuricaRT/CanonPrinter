\chapter{Arhitektura i dizajn sustava}

		\text{Arhitektura aplikacije može se podijeliti na 3 podsustava:}
	\begin{itemize}
		\item 	\text{Web preglednik}
		\item 	\text{Web poslužitelj/Web aplikacija}
		\item 	\text{Baza podataka}		
	\end{itemize}

	\begin{figure}[H]
		\centering
		\includegraphics[width=100mm]{slike/arhitektura.png}
		\caption{Arhitektura sustava}
		\label{fig:arhitektura}
	\end{figure}

	\indent{\textit{\underline{Web preglednik}} je program koji korisniku omogućuje pregled
	web-stranica i multimedijalnih sadržaja vezanih uz njih. Svaki web preglednik je predvoditelj korištenja web aplikacija,
	jer omogućuje korisniku da preko web preglednika šalje zahtjeve web poslužitelju.}

	\indent{\textit{\underline{Web poslužitelj}} je osnova rada web aplikacije. On pokreće cijeli sustav rada aplikacije
	te joj prosljeđuje zahtjeve od korisnika. Osnovna zadaća web poslužitelja je omogućiti komunikaciju između korisnika i aplikacije, a ta
	komunikacija se odvija preko HTTP protokola. To je vrsta protokola koja se koristi za prijenos informacija na internetu.}

	\indent{\textit{\underline{Web aplikacija}} je dio web poslužitelja koja služi korisniku za obradu željenih zahtjeva.
	Web aplikacija radi tako da prima zahtjeve i ovisno o zahtjevu pristupa {\textit{\underline{bazi podataka}}} iz koje dohvaća
	"odgovore" na željene zahtjeve. Te "odgovore" šalje natrag korisniku preko web poslužitelja u obliku HTML dokumenta kojeg
	korisnik vidi u web pregledniku.}

	\indent{Programski jezik kojeg smo odabrali za izradu naše web aplikacije je Python. U sklopu Pythona koristimo Django, radni okvir koji služi za izradu web aplikacija. Razvojno okruženje koje koristimo
	je Microsoft Visual Studio Code. Arhitektura sustava temelji se na MVC odnosno MTV konceptu.}

	\indent{Django je modeliran oko MVC arhitekture, no svoju arhitekturu definira kao MTV (eng. {\textit{Model-Template-View}})
	arhitekturu. Komponentu upravitelj (eng. {\textit{Controller}}) zamjenjuje komponentom pogled (eng. {\textit{View}}) te
	komponentu pogled s komponentom predložak (eng. {\textit{Template}}). MTV razdvaja različite dijelove web-stranice: prikaz,
	pristup podatcima i logiku web stranice. Također omogućava neovisnu izgradnju web-stranica, povećava sigurnost sustava te
	pojednostavljuje održavanje sustava.}

	\noindent{MTV se sastoji od:}
	\begin{itemize}
		\item 	\textbf{Model} - definira oblike i odnose podataka u bazi podataka. Model u Django okruženju je klasa napisana
				u programskom jeziku Python. Određuje varijable i metode pridužene određenim tipovima podataka te ima značenje
				tablice u bazi podataka. Model je usko povezan s bazom podataka i pogledom. Od baze podataka model dohvaća tražene
				podatke i prosljeđuje ih pogledu.
		\item 	\textbf{Predložak} - sloj arhitekture MTV-a usko povezan s web-preglednikom. Predložak je HTML stranica s dodanim
				strukturama koje omogućavaju prikaz podataka koji su proslijeđeni od pogleda. Zadaća predloška je sadržaj primljen
				od pogleda organizirati i ugraditi u HTML kod koji će se prikazati u web-pregledniku.
		\item 	\textbf{Pogled} - određuje koji će podatci biti prikazani, odnosno, koji će podatci biti dohvaćeni iz baze podataka
				i prikazani pomoću predloška u web-pregledniku. U Djangu prilikom stvaranja nove web-aplikacije za svaku pojedinu
				aplikaciju stvara se zasebna datoteka pogleda. Pogled ne zna kako su podatci prikazani u web-pregledniku. Posao
				pogleda je dohvatiti tražene podatke i proslijediti ih višem sloju koji će ih prikazati u pregledniku.
	\end{itemize}
	
		

		

				
		\section{Baza podataka}
			
			\textbf{\textit{dio 1. revizije}}\\
			
		\textit{Potrebno je opisati koju vrstu i implementaciju baze podataka ste odabrali, glavne komponente od kojih se sastoji i slično.}
		
			\subsection{Opis tablica}
			

				\textit{Svaku tablicu je potrebno opisati po zadanom predlošku. Lijevo se nalazi točno ime varijable u bazi podataka, u sredini se nalazi tip podataka, a desno se nalazi opis varijable. Svjetlozelenom bojom označite primarni ključ. Svjetlo plavom označite strani ključ}
				
				
				\begin{longtblr}[
					label=none,
					entry=none
					]{
						width = \textwidth,
						colspec={|X[6,l]|X[6, l]|X[20, l]|}, 
						rowhead = 1,
					} %definicija širine tablice, širine stupaca, poravnanje i broja redaka naslova tablice
					\hline \SetCell[c=3]{c}{\textbf{korisnik - ime tablice}}	 \\ \hline[3pt]
					\SetCell{LightGreen}IDKorisnik & INT	&  	Lorem ipsum dolor sit amet, consectetur adipiscing elit, sed do eiusmod  	\\ \hline
					korisnickoIme	& VARCHAR &   	\\ \hline 
					email & VARCHAR &   \\ \hline 
					ime & VARCHAR	&  		\\ \hline 
					\SetCell{LightBlue} primjer	& VARCHAR &   	\\ \hline 
				\end{longtblr}
				
				
			
			\subsection{Dijagram baze podataka}
				\textit{ U ovom potpoglavlju potrebno je umetnuti dijagram baze podataka. Primarni i strani ključevi moraju biti označeni, a tablice povezane. Bazu podataka je potrebno normalizirati. Podsjetite se kolegija "Baze podataka".}
			
			\eject
			
			
		\section{Dijagram razreda}
		
			\textit{Potrebno je priložiti dijagram razreda s pripadajućim opisom. Zbog preglednosti je moguće dijagram razlomiti na više njih, ali moraju biti grupirani prema sličnim razinama apstrakcije i srodnim funkcionalnostima.}\\
			
			\textbf{\textit{dio 1. revizije}}\\
			
			\textit{Prilikom prve predaje projekta, potrebno je priložiti potpuno razrađen dijagram razreda vezan uz \textbf{generičku funkcionalnost} sustava. Ostale funkcionalnosti trebaju biti idejno razrađene u dijagramu sa sljedećim komponentama: nazivi razreda, nazivi metoda i vrste pristupa metodama (npr. javni, zaštićeni), nazivi atributa razreda, veze i odnosi između razreda.}\\
			
			\textbf{\textit{dio 2. revizije}}\\			
			
			\textit{Prilikom druge predaje projekta dijagram razreda i opisi moraju odgovarati stvarnom stanju implementacije}
			
			
			
			\eject
		
		\section{Dijagram stanja}
			
			
			\textbf{\textit{dio 2. revizije}}\\
			
			\textit{Potrebno je priložiti dijagram stanja i opisati ga. Dovoljan je jedan dijagram stanja koji prikazuje \textbf{značajan dio funkcionalnosti} sustava. Na primjer, stanja korisničkog sučelja i tijek korištenja neke ključne funkcionalnosti jesu značajan dio sustava, a registracija i prijava nisu. }
			
			
			\eject 
		
		\section{Dijagram aktivnosti}
			
			\textbf{\textit{dio 2. revizije}}\\
			
			 \textit{Potrebno je priložiti dijagram aktivnosti s pripadajućim opisom. Dijagram aktivnosti treba prikazivati značajan dio sustava.}
			
			\eject
		\section{Dijagram komponenti}
		
			\textbf{\textit{dio 2. revizije}}\\
		
			 \textit{Potrebno je priložiti dijagram komponenti s pripadajućim opisom. Dijagram komponenti treba prikazivati strukturu cijele aplikacije.}