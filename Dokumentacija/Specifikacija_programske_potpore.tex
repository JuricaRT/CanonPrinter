\chapter{Specifikacija programske potpore}
		
	\section{Funkcionalni zahtjevi}
			
			\noindent \textbf{Dionici:}
			
			\begin{packed_enum}
				
				\item Naručitelji
				\item Učenici (korisnici)
				\item Administratori				
				\item Razvojni tim
				
			\end{packed_enum}
			
			\noindent \textbf{Aktori i njihovi funkcionalni zahtjevi:}
			
			
			\begin{packed_enum}
				\item  \underbar{Neregistrirani učenik (inicijator) može:}
				
				\begin{packed_enum}
					
					\item se registrirati adresom elektroničke pošte
					
				\end{packed_enum}
			
			
				\item  \underbar{Učenik (inicijator) može:}
				
				\begin{packed_enum}
					\item izvršiti prijavu u sustav za koju su mu potrebni adresa elektroničke pošte i lozinka
					\item promijeniti trenutnu lozinku
					\item obrisati svoj korisnički račun
					\item pregledavati postojeće rječnike grupirane po jeziku
					\item pokrenuti učenje riječi odabirom jednog od ponuđenih rječnika i načina učenja
					\item odgovarati na pitanja o riječima ovisno o odabranom načinu učenja
					
					\begin{packed_enum}
						\item odabirom točnog odgovora
						\item upisivanjem točnog odgovora
						\item snimanjem izgovora riječi u zvučnu datoteku
					\end{packed_enum}
					
				\end{packed_enum}
			
					
			\item  \underbar{Administrator (inicijator) može:}
			
			\begin{packed_enum}
				
				\item stvarati nove rječnike
				\item dodavati i brisati riječi iz rječnika
				\item uređivati komponente postojećih riječi u rječniku
				
			\end{packed_enum}
			
			
			\item  \underbar{Korijenski administrator (inicijator) može:}
			
			\begin{packed_enum}
				
				\item definirati administratore
				\item stvarati nove rječnike
				\item dodavati i brisati riječi iz rječnika
				\item uređivati komponente postojećih riječi u rječniku
				
			\end{packed_enum}
		
			
			\item  \underbar{Baza podataka (sudionik):}
		
			\begin{packed_enum}
			
			\item pohranjuje rječnike
			\item pohranjuje sve podatke o učenicima i administratorima
			
			\end{packed_enum}
		
			
			\item  \underbar{Vanjski rječnik (sudionik):}
		
			\begin{packed_enum}
			
			\item sadrži informacije koje se koriste prilikom dodavanja novih riječi u rječnike
			
			\end{packed_enum}
		
			
			\item  \underbar{Servis za ocjenu kvalitete izgovora (sudionik):}
	
			\begin{packed_enum}
		
			\item provjerava točnost snimljene izgovorene riječi
			\item na temelju provjere odgovara s ocjenom u ljestvici od jedan do deset
		
			\end{packed_enum}		
		
	
			\end{packed_enum}
			
			\eject 
			
			
				
			\subsection{Obrasci uporabe}
				
				\textbf{\textit{dio 1. revizije}}
				
				\subsubsection{Opis obrazaca uporabe}
					\textit{Funkcionalne zahtjeve razraditi u obliku obrazaca uporabe. Svaki obrazac je potrebno razraditi prema donjem predlošku. Ukoliko u nekom koraku može doći do odstupanja, potrebno je to odstupanje opisati i po mogućnosti ponuditi rješenje kojim bi se tijek obrasca vratio na osnovni tijek.}\\
					

					\noindent \underbar{\textbf{UC$<$1$>$ -$<$Registracija$>$}}
					\begin{packed_item}
	
						\item \textbf{Glavni sudionik: }$<$Učenik$>$
						\item  \textbf{Cilj:} $<$Stvoriti korisnički račun za pristup sustavu$>$
						\item  \textbf{Sudionici:} $<$Baza podataka$>$
						\item  \textbf{Preduvjet:} $<$-$>$
						\item  \textbf{Opis osnovnog tijeka:}
						
						\item[] \begin{packed_enum}
	
							\item $<$Učenik odabire gumb za registraciju$>$
							\item $<$Učenik unosi potrebne korisničke podatke$>$
							\item $<$Učenik prima obavijest o uspješnoj registraciji$>$
						\end{packed_enum}
						
						\item  \textbf{Opis mogućih odstupanja:}
						
						\item[] \begin{packed_item}
	
							\item[2.a] $<$Unos krivih korisničkih podatka, odabir već zauzetog e-maila, unos već zauzetog korisničkog imena, unos krivog e-maila$>$
							\item[] \begin{packed_enum}
								
								\item $<$Sustav obavještava učenika o neuspjeloj registraciji i vraća ga natrag na stranicu za registraciju$>$
								\item $<$Učenik mijenja potrebne unesene podatke i završava unos ili odustaje od registracije$>$
								
							\end{packed_enum}
							
						\end{packed_item}
					\end{packed_item}

					\noindent \underbar{\textbf{UC$<$2$>$ -$<$Prijava u sustav$>$}}
					\begin{packed_item}
	
						\item \textbf{Glavni sudionik: }$<$Učenik$>$
						\item  \textbf{Cilj:} $<$Dobiti pristup sučelju za učenje stranih jezika$>$
						\item  \textbf{Sudionici:} $<$Baza podataka$>$
						\item  \textbf{Preduvjet:} $<$Registracija$>$
						\item  \textbf{Opis osnovnog tijeka:}
						
						\item[] \begin{packed_enum}
	
							\item $<$Unos korisničkog imena ili e-maila i lozinke$>$
							\item $<$Potvrda o ispravnosti unesenih podataka$>$
							\item $<$Pristup korisničkom sučelju$>$
						\end{packed_enum}
						
						\item  \textbf{Opis mogućih odstupanja:}
						
						\item[] \begin{packed_item}
	
							\item[2.a] $<$Neispravno uneseno korisničko ime/lozinka$>$
							\item[] \begin{packed_enum}
								
								\item $<$Sustav obavještava učenika o neuspjeloj prijavi te ga vraća na stranicu za prijavu$>$
								
							\end{packed_enum}
							
						\end{packed_item}
					\end{packed_item}

					\noindent \underbar{\textbf{UC$<$3$>$ -$<$Pregled osobnih podataka$>$}}
					\begin{packed_item}
	
						\item \textbf{Glavni sudionik: }$<$Učenik$>$
						\item  \textbf{Cilj:} $<$Pregledati osobne podatke$>$
						\item  \textbf{Sudionici:} $<$Baza podataka$>$
						\item  \textbf{Preduvjet:} $<$Učenik je prijavljen u sustav$>$
						\item  \textbf{Opis osnovnog tijeka:}
						
						\item[] \begin{packed_enum}
	
							\item $<$Učenik odlazi na svoj profil$>$
							\item $<$Učenik odabire opciju "Osobni podatci"$>$
							\item $<$Aplikacije prikazuje osobne podatke učenika$>$
						\end{packed_enum}
						
					\end{packed_item}

					\noindent \underbar{\textbf{UC$<$4$>$ -$<$Promjena osobnih podataka$>$}}
					\begin{packed_item}
	
						\item \textbf{Glavni sudionik: }$<$Učenik$>$
						\item  \textbf{Cilj:} $<$Promjeniti osobne podatke$>$
						\item  \textbf{Sudionici:} $<$Baza podataka$>$
						\item  \textbf{Preduvjet:} $<$Učenik je prijavljen u sustav$>$
						\item  \textbf{Opis osnovnog tijeka:}
						
						\item[] \begin{packed_enum}
	
							\item $<$Učenik odlazi na svoj profil$>$
							\item $<$Učenik bira koje podatke želi mijenjati$>$
							\item $<$Učenik mijenja određene osobne podatke$>$
							\item $<$Učenik sprema promjene$>$
							\item $<$Baza podataka ažurira napravljenje promjene$>$
						\end{packed_enum}
						
						\item  \textbf{Opis mogućih odstupanja:}
						
						\item[] \begin{packed_item}
	
							\item[2.a] $<$Učenik mijenja svoje podatke, ali ne spremi promjene$>$
							\item[] \begin{packed_enum}
								
								\item $<$Sustav obavještava učenika da nije spremio podatke$>$
								\item $<$Učenik odabire želi li spremiti napravljenje promjene ili ostaviti podatke kakvi su bili$>$
								
							\end{packed_enum}
							
						\end{packed_item}
					\end{packed_item}

					\noindent \underbar{\textbf{UC$<$5$>$ -$<$Brisanje korisničkog računa$>$}}
					\begin{packed_item}
	
						\item \textbf{Glavni sudionik: }$<$Učenik$>$
						\item  \textbf{Cilj:} $<$Izbrisati svoj korisnički račun$>$
						\item  \textbf{Sudionici:} $<$Baza podataka$>$
						\item  \textbf{Preduvjet:} $<$Učenik je prijavljen u sustav$>$
						\item  \textbf{Opis osnovnog tijeka:}
						
						\item[] \begin{packed_enum}
	
							\item $<$Učenik odlazi na svoj profil$>$
							\item $<$Učenik bira opciju obriši korisnički račun$>$
							\item $<$Sustav ispituje učenika je li siguran da želi obrisati korisnički račun$>$
							\item $<$Učenik potvrđuje da želi obrisati korisnički račun$>$
							\item $<$Korisnički račun se briše iz baze podataka$>$
							\item $<$Sustav šalje učenika na stranicu za registraciju$>$
						\end{packed_enum}
						
						\item  \textbf{Opis mogućih odstupanja:}
						
						\item[] \begin{packed_item}
	
							\item[2.a] $<$Učenik ne potvrđuje brisanje računa, a izlazi iz aplikacije$>$
							\item[] \begin{packed_enum}
								
								\item $<$Sustav ne briše korisnički račun i obavještava učenika da korisnički račun nije obrisan$>$
								
							\end{packed_enum}
							
						\end{packed_item}
					\end{packed_item}

					\noindent \underbar{\textbf{UC$<$broj obrasca$>$ -$<$ime obrasca$>$}}
					\begin{packed_item}
	
						\item \textbf{Glavni sudionik: }$<$sudionik$>$
						\item  \textbf{Cilj:} $<$cilj$>$
						\item  \textbf{Sudionici:} $<$sudionici$>$
						\item  \textbf{Preduvjet:} $<$preduvjet$>$
						\item  \textbf{Opis osnovnog tijeka:}
						
						\item[] \begin{packed_enum}
	
							\item $<$opis korak jedan$>$
							\item $<$opis korak dva$>$
							\item $<$opis korak tri$>$
							\item $<$opis korak četiri$>$
							\item $<$opis korak pet$>$
						\end{packed_enum}
						
						\item  \textbf{Opis mogućih odstupanja:}
						
						\item[] \begin{packed_item}
	
							\item[2.a] $<$opis mogućeg scenarija odstupanja u koraku 2$>$
							\item[] \begin{packed_enum}
								
								\item $<$opis rješenja mogućeg scenarija korak 1$>$
								\item $<$opis rješenja mogućeg scenarija korak 2$>$
								
							\end{packed_enum}
							\item[2.b] $<$opis mogućeg scenarija odstupanja u koraku 2$>$
							\item[3.a] $<$opis mogućeg scenarija odstupanja  u koraku 3$>$
							
						\end{packed_item}
					\end{packed_item}
				
					
				\subsubsection{Dijagrami obrazaca uporabe}
					
					\textit{Prikazati odnos aktora i obrazaca uporabe odgovarajućim UML dijagramom. Nije nužno nacrtati sve na jednom dijagramu. Modelirati po razinama apstrakcije i skupovima srodnih funkcionalnosti.}
				\eject		
				
			\subsection{Sekvencijski dijagrami}
				
				\textbf{\textit{dio 1. revizije}}\\
				
				\textit{Nacrtati sekvencijske dijagrame koji modeliraju najvažnije dijelove sustava (max. 4 dijagrama). Ukoliko postoji nedoumica oko odabira, razjasniti s asistentom. Uz svaki dijagram napisati detaljni opis dijagrama.}
				\eject
	
		\section{Ostali zahtjevi}
		
			\textbf{\textit{dio 1. revizije}}\\
		 
			 \textit{Nefunkcionalni zahtjevi i zahtjevi domene primjene dopunjuju funkcionalne zahtjeve. Oni opisuju \textbf{kako se sustav treba ponašati} i koja \textbf{ograničenja} treba poštivati (performanse, korisničko iskustvo, pouzdanost, standardi kvalitete, sigurnost...). Primjeri takvih zahtjeva u Vašem projektu mogu biti: podržani jezici korisničkog sučelja, vrijeme odziva, najveći mogući podržani broj korisnika, podržane web/mobilne platforme, razina zaštite (protokoli komunikacije, kriptiranje...)... Svaki takav zahtjev potrebno je navesti u jednoj ili dvije rečenice.}
			 
			 
			 
	