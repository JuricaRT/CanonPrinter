\chapter{Dnevnik promjena dokumentacije}
		
		\begin{longtblr}[
				label=none
			]{
				width = \textwidth, 
				colspec={|X[2]|X[13]|X[3]|X[3]|}, 
				rowhead = 1
			}
			\hline
			\textbf{Rev.}	& \textbf{Opis promjene/dodatka} & \textbf{Autori} & \textbf{Datum}\\[3pt] \hline
			0.1 & Napravljen predložak. \newline Upisane osnovne informacije o timu. & Jurica Runtas & 22.10.2023. \\[3pt] \hline 
			0.2	& Dodani dionici, aktori i njihovi funkcionalni zahtjevi. & Jurica Runtas & 25.10.2023. 	\\[3pt] \hline 
			0.3 & Opisi obrazaca uporabe & Lovro Švenda & 27.10.2023. \\[3pt] \hline 
			0.3.1 & Dodani dijagrami obrasca uporabe, izmjena opisa obrasca uporabe & Kristijan Milić & 29.10.2023. \\[3pt] \hline 
			0.3.2 & Uređen format opisa obrazaca uporabe i napravljene manje izmjene & Jurica Runtas & 29.10.2023. \\[3pt] \hline 
			0.4 & Sekvencijski dijagrami & Matej Galić & 30.10.2023. \\[3pt] \hline
			0.4.1 & Napravljeni prvi i drugi sekvencijski dijagrami & Matej Galić & 30.10.2023. \\[3pt] \hline
			0.4.2 & Napravljeni treći i četvrti sekvencijski dijagrami & Josip Ćurić & 30.10.2023. \\[3pt] \hline
		\end{longtblr}
	
	
		\textit{Moraju postojati glavne revizije dokumenata 1.0 i 2.0 na kraju prvog i drugog ciklusa. Između tih revizija mogu postojati manje revizije već prema tome kako se dokument bude nadopunjavao. Očekuje se da nakon svake značajnije promjene (dodatka, izmjene, uklanjanja dijelova teksta i popratnih grafičkih sadržaja) dokumenta se to zabilježi kao revizija. Npr., revizije unutar prvog ciklusa će imati oznake 0.1, 0.2, …, 0.9, 0.10, 0.11.. sve do konačne revizije prvog ciklusa 1.0. U drugom ciklusu se nastavlja s revizijama 1.1, 1.2, itd.}